\documentclass{article}
\usepackage{german}
\usepackage[latin1]{inputenc}
\usepackage{a4wide}
\usepackage{epsfig}
\usepackage{amssymb}
\usepackage{fancyvrb}
\usepackage{alltt}
\usepackage{fleqn}
\usepackage{epic}

\newtheorem{Definition}{Definition}
\newtheorem{Axiom}[Definition]{Axiom}
\newtheorem{Notation}[Definition]{Notation}
\newtheorem{Korollar}[Definition]{Korollar}
\newtheorem{Lemma}[Definition]{Lemma}
\newtheorem{Satz}[Definition]{Satz}
\newtheorem{Theorem}[Definition]{Theorem}

\renewcommand{\labelenumi}{(\alph{enumi})}

\title{Analysis}
\author{Karl Stroetmann}

\newcommand{\bruch}[2]{\displaystyle\frac{\;\displaystyle#1\;}{\;\displaystyle#2\;}}
\newcommand{\folge}[1]{\bigl(#1\bigr)_{n\in\mathbb{N}}}
\newcommand{\Folge}[1]{\left(#1\right)_{n\in\mathbb{N}}}
\newcommand{\Reihe}[1]{\left(\sum\limits_{i=0}^n #1\right)_{n\in\mathbb{N}}}
\newcommand{\Oh}{\mathcal{O}}
\newcommand{\df}{\displaystyle\frac{d\;}{dx}}
\newcommand{\err}[1]{\textsl{error}_n(#1)}
\newcommand{\erri}[2]{\textsl{error}^{(#2)}_n(#1)}
\newcommand{\norm}[1]{\big\|#1\bigr\|_{\infty}}

\def\pair(#1,#2){\langle #1, #2 \rangle}

\newlength{\mylength}
\setlength{\mathindent}{1.3cm}


\begin{document}
\noindent
\section{Taylor-Reihe}
\textbf{Aufgabe}: Leiten Sie die folgende Formel aus dem Additions-Theorem des
Arcus-Tangens her und berechnen Sie damit $\pi$ auf eine Genauigkeit von $10^{-9}$:
\begin{equation}
  \label{eq:Pi}
  \bruch{\pi}{4} =  2 * \arctan\Bigl(\frac{1}{2}\Bigr) -\arctan\Bigl(\frac{1}{7}\Bigr).  
\end{equation}


\section{Interpolation}
\textbf{Aufgabe}: F�r die Funktion $x \mapsto \sin(x)$ soll im Intervall
$[0,\frac{\pi}{2}]$ ein Tabelle erstellt werden, so dass der bei linearer Interpolation entstehende
Interpolations-Fehler kleiner als $10^{-5}$ ist.  Das Intervall $[0,\frac{\pi}{2}]$ soll zu diesem
Zweck in gleich gro�e Intervalle aufgeteilt werden.  Berechnen Sie die Anzahl der
Eintr�ge, die f�r die Erstellung der Tabelle notwendig ist.
\vspace*{0.3cm}


\noindent
\textbf{Aufgabe}: L�sen Sie f�r $y=10^6$ und $y=10^{-6}$ die Gleichung $x*\exp(x) = y$ 
durch eine einfache Fixpunkt-Iteration.  Berechnen Sie die L�sung $x$ jeweils auf eine
Genauigkeit von $10^{-3}$.
\vspace*{0.3cm}


\noindent
\textbf{Aufgabe}: Untersuchen Sie mit Hilfe des Integral-Vergleichskriteriums, ob die
Reihe 
\\[0.1cm]
\hspace*{1.3cm}
$\displaystyle \sum\limits_{n=1}^\infty \bruch{1}{n*(n+1)}$ 
\\[0.3cm]
konvergiert.
\vspace*{0.3cm}
\pagebreak

\noindent
\textbf{Aufgabe}:  Berechnen Sie, in wieviele Teil-Intervalle das Intervall $[0,1]$
aufgeteilt werden muss, wenn das Integral
\\[0.1cm]
\hspace*{1.3cm}
$\displaystyle \int_0^1 e^{-x^2}\, dx$
\\[0.1cm]
mit Hilfe der Trapez-Regel mit einer Genauigkeit von $10^{-6}$ berechnet werden soll.
\vspace*{0.3cm}

\noindent
\textbf{Aufgabe}: 
\begin{enumerate}
\item Berechnen Sie mit Hilfe der Kepler'schen Fass-Regel eine Approximation 
      f�r das Integral 
      \\[0.1cm]
      \hspace*{1.3cm}$\displaystyle \int_0^{\frac{1}2} \sin(x)\, dx$.
\item Geben Sie eine m�glichst genaue Absch�tzung f�r den Approximations-Fehler.
\item Vergleichen Sie ihr Ergebnis mit dem exakten Wert.
\end{enumerate}
\vspace*{0.3cm}


\noindent
\textbf{Aufgabe}: Gegenstand dieser Aufgabe ist die numerische Berechnung der Summe 
\\[0.1cm]
\hspace*{1.3cm} $\displaystyle \sum\limits_{k=1}^\infty \bruch{1}{k^3}$.
\\[0.1cm]
Gehen Sie zur Berechnung dieser Summe in folgenden Schritten vor.
\begin{enumerate}
\item Approximieren Sie die Rest-Summe $\sum\limits_{k=n}^\infty \bruch{1}{k^3}$ 
      durch ein geeignetes Integral.

      \textbf{Hinweis}: Es gilt 
      \\[0.1cm]
      \hspace*{1.3cm} $\displaystyle f(k) = \int_{k-\frac{1}{2}}^{k+\frac{1}{2}} f(k) \, dt \approx \int_{k-\frac{1}{2}}^{k+\frac{1}{2}} f(t) \, dt$.
\item Berechnen Sie eine Absch�tzung f�r den Approximations-Fehler,
      den Sie bei der Integration in Teil (a) erhalten.
      
      \textbf{Hinweis}: Approximieren Sie die auftretenden Summen durch Integrale.
\item Berechnen Sie nun, wir gro� Sie $n$ w�hlen m�ssen, damit der Approximations-Fehler
      kleiner als $10^{-6}$ bleibt.
\item Geben Sie nun einen N�herungs-Wert f�r die Summe $\sum\limits_{k=1}^\infty \bruch{1}{k^3}$,
      der sich von dem exakten Ergebnis um weniger als $10^{-6}$ unterscheidet.
\end{enumerate}
\vspace*{0.3cm} 

\noindent
\textbf{Aufgabe}:
Die Funktion $p$ sei auf dem Intervall $[-\pi,\pi]$ definiert durch
\\[0.1cm]
\hspace*{1.3cm}
$p(x) = x^2$.
\\[0.1cm]
Die Funktion werde so auf $\mathbb{R}$ fortgesetzt, dass die resultierende Funktion die Periode
$2\!\cdot\!\pi$ hat.  
\begin{enumerate}
\item Berechnen Sie die Fourier-Reihe von $p$.
\item Berechnen Sie mit Hilfe der Fourier-Reihe von $p$ einen Wert f�r die Summe
      \\[0.1cm]
      \hspace*{1.3cm}
      $\displaystyle \sum\limits_{n=1}^\infty \bruch{1}{n^2}$. 
\end{enumerate}

\pagebreak
\noindent
\textbf{Aufgabe}:  Die Lambert'sche W-Funktion (Johann Heinrich Lambert; 1728 - 1777)
$x \mapsto W(x)$ ist f�r $x\geq 0$ definiert als die Umkehr-Funktion der Funktion $x \mapsto x\cdot e^x$,
es gilt also 
\\[0.1cm]
\hspace*{1.3cm} $\displaystyle W(x) \cdot e^{W(x)} = x$ \quad f�r alle $x \in \mathbb{R}_+$.
\begin{enumerate}
\item Berechnen Sie die Ableitung der Lambert'schen W-Funktion.
\item Formen Sie den Ausdruck f�r $W'(x)$ so um, dass der Term $e^{W(x)}$ nicht mehr
      auftritt.
\item Berechnen Sie eine Stamm-Funktion der Lambert'schen W-Funktion.
\item Nehmen Sie an, dass Sie die Lambert'sche W-Funktion berechnen k�nnen und
      bestimmen Sie unter dieser Annahme f�r ein gegebenes $\varepsilon$ die L�sung der Gleichung 
      \\[0.1cm]
      \hspace*{1.3cm}
      $\bruch{1}{n} * \Bigl(\frac{1}{2}\Bigr)^n = \varepsilon$
      \\[0.1cm]
      durch algebraische Umformungen.

      \noindent
      \textbf{Hinweis}: Invertieren Sie die Gleichung und bringen Sie die Gleichung dann
      auf die Form $\alpha * e^\alpha = \beta$ f�r geeignete $\alpha$ und $\beta$, denn
      dann gilt $\alpha = W(\beta)$.
\end{enumerate}
\vspace*{0.3cm}


\noindent
 \textbf{Aufgabe}: Zeigen Sie anhand der Definition des Grenzwerts, dass 
      $\lim\limits_{n \rightarrow \infty} \bruch{1}{\sqrt{n}} = 0$ ist.
%\item Zeigen Sie, dass die Folge $\folge{\sin(n)}$ nicht konvergent ist.
%      leider zu schwer
\vspace*{0.3cm}

\noindent
\textbf{Aufgabe}:
Es seien $a,b\in\mathbb{R}$ gegeben. 
Die Folge $\folge{c_n}$ werde induktiv definiert durch $c_0 = a$, $c_1 = b$ und  
\\[0.1cm]
\hspace*{1.3cm} $c_{n+2} = \frac{1}{2}*(c_n + c_{n+1})$.
\begin{enumerate}
\item Zeigen Sie durch Induktion nach $n$, dass gilt: 
      \\[0.1cm]
      \hspace*{1.3cm}
      $c_n = \bruch{1}{3}\cdot a \biggl(1-\Bigl(\frac{-1}{2}\Bigr)^{n-1}\biggr) + \bruch{2}{3}\cdot b \biggl(1-\Bigl(\frac{-1}{2}\Bigr)^{n}\biggr)$
\item Berechnen Sie den Grenzwert $\lim\limits_{n \rightarrow \infty} c_n$.
\end{enumerate}

\end{document}





%%% Local Variables: 
%%% mode: latex
%%% TeX-master: t
%%% End: 
