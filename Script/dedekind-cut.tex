\chapter{Die Definition der reellen Zahlen}
Bevor wir mit der eigentlichen Analysis beginnen m�ssen wir kl�ren, was genau reelle Zahlen
�berhaupt sind.  Anschaulich werden reelle Zahlen zur Angabe von L�ngen ben�tigt, denn in der Geometrie
reicht es nicht, mit den rationalen Zahlen zu arbeiten.  Das liegt daran, dass die Diagonale eines
Quadrats der Seitenl�nge 1 nach dem 
\href{http://de.wikipedia.org/wiki/Satz_des_Pythagoras}{Satz des Pythagoras} die L�nge $\sqrt{2}$ hat.
Wir hatten im letzten Semester aber gesehen, dass es keine rationale Zahl $r$ gibt,
so dass $r^2 = 2$ ist.  Folglich reichen die rationalen Zahlen nicht aus, alle in der Geometrie
m�glichen L�ngen anzugeben. 

Bis jetzt haben wir so getan, als w�ssten wir schon, was reelle Zahlen sind.  Aus der Schule bringen
Sie gewiss eine anschauliche Vorstellung der reellen Zahlen mit, aber diese Vorstellung gilt es
nun zu formalisieren, denn sonst k�nnen wir den Nachweis der \emph{Vollst�ndigkeit} der reellen
Zahlen nicht f�hren.   Unter der Vollst�ndigkeit der reellen Zahlen verstehen wir anschaulich die Eigenschaft,
dass es auf  der reellen Zahlengeraden keine L�cher gibt.  Bei der formalen Darstellung der reellen
Zahlen hilft uns die folgende Definition.

\begin{Definition}[Dedekind-Schnitt] \lb
Ein Paar $\pair(M_1,M_2)$ hei�t \href{http://de.wikipedia.org/wiki/Dedekindscher_Schnitt}{\emph{Dedekind-Schnitt}}
(\href{http://de.wikipedia.org/wiki/Richard_Dedekind}{\textrm{Richard Dedekind}}, 1831-1916)
falls folgendes gilt:
\begin{enumerate}
\item $M_1 \subseteq  \mathbb{Q}$, \quad $M_2 \subseteq \mathbb{Q}$.
\item $M_1 \not= \emptyset$, \quad $M_2 \not= \emptyset$.
\item $\forall x_1 \in M_1: \forall x_2 \in M_2: x_1 < x_2$.
\item $M_1 \cup M_2 = \mathbb{Q}$.
\end{enumerate}
\end{Definition}

\example Definieren wir
\\[0.2cm]
\hspace*{1.3cm} 
$M_1 := \{ x \in \mathbb{Q} \mid x \leq 0 \vee x^2 \leq 2 \}$ \quad und \quad
$M_2 := \{ x \in \mathbb{Q} \mid x > 0 \wedge x^2 > 2 \}$,
\\[0.2cm]
so enth�lt $M_1$ alle die Zahlen, die kleiner oder gleich $\sqrt{2}$ sind, w�hrend
$M_2$ alle Zahlen enth�lt, die gr��er als $\sqrt{2}$ sind. Das Paar $\pair(M_1,M_2)$ ist dann ein
Dedekind-Schnitt. \eox

\noindent
Formal ist die Menge der reellen Zahlen als die Menge aller Dedekind-Schnitte definiert:
\\[0.2cm]
\hspace*{1.3cm}
\framebox{
\framebox{
$\mathbb{R} := \bigl\{ \pair(M_1,M_2) \in 2^\mathbb{Q} \times 2^\mathbb{Q} \mid \mbox{$\pair(M_1,M_2)$ ist eine Dedekind-Schnitt}\bigr\}$.}}
\\[0.2cm]
Nach dieser Definition m�ssen wir nun zeigen, wie sich auf der so definierten Menge der reellen
Zahlen die arithmetischen Operationen Addition, Subtraktion, Multiplikation und Division definieren
lassen.  Zus�tzlich ist zu kl�ren, wie die Relation $<$ f�r zwei Dedekind-Schnitte definiert
werden kann.  Die folgende Definition der Dedekind-Menge ist dabei hilfreich.
\pagebreak

\begin{Definition}[Dedekind-Menge]
Eine Menge $M \subseteq \mathbb{Q}$ ist eine \emph{Dedekind-Menge} genau dann, wenn die
folgenden Bedingungen erf�llt sind.
\begin{enumerate}
\item $M \not= \{\}$,
\item $M \not= \mathbb{Q}$,
\item $\forall x, y \in \mathbb{Q}: \bigl(y < x \wedge x \in M \rightarrow y \in M)$.

      Die letzte Bedingung besagt, dass $M$ \emph{nach unten abgeschlossen} ist:  Wenn eine
      Zahl $x$ in $M$ liegt, dann liegt auch jede Zahl, die kleiner als $x$ ist, in $M$.
\item Die Menge $M$ hat kein Maximum, es gibt also kein $m \in M$, so dass
      \\[0.2cm]
      \hspace*{1.3cm}
      $x \leq m$ \quad f�r alle $x \in M$ gilt.
\end{enumerate}
\end{Definition}

\exercise
Zeigen Sie, dass eine Menge $M \subseteq \mathbb{Q}$ genau dann eine Dedekind-Menge ist, wenn das Paar $\pair(M,\mathbb{Q} \backslash M)$
ein Dedekind-Schnitt ist. \eox

\exercise
Es sei $\mathcal{D}$ die Menge aller rationalen Dedekind-Mengen, also
\\[0.2cm]
\hspace*{1.3cm}
$\mathcal{D} := \bigl\{ M \in 2^{\mathbb{Q}} \mid \mbox{$M$ is Dedekind-Menge} \bigr\}$.
\\[0.2cm]
Auf der Menge $\mathcal{D}$ definieren wir eine bin�re Relation $\leq$ durch die Festsetzung
\\[0.2cm]
\hspace*{1.3cm}
$A \leq B \;\stackrel{\mathrm{def}}{\Longleftrightarrow}\; A \subseteq B$ \quad f.a. $A,B \in \mathcal{D}$.
\\[0.2cm]
Zeigen Sie, dass die so definierte Relation $\leq$ eine totale Ordnung auf der Menge $\mathcal{D}$ ist.

\begin{Definition}[Infimum]
Eine Menge $\mathcal{M} \subseteq \mathcal{D}$ ist \emph{in $\mathcal{D}$ nach unten beschr�nkt}, falls es ein
 $U \in \mathcal{D}$ gibt, so dass gilt:
\\[0.2cm]
\hspace*{1.3cm}
$\forall A \in \mathcal{M}:  U \leq A$.
\\[0.2cm]
Eine Menge $I \in \mathcal{D}$ ist das \emph{Infimum} einer Menge $\mathcal{M}$, wenn $I$ die gr��te
untere Schranke von $\mathcal{M}$ ist, wenn also 
\\[0.2cm]
\hspace*{1.3cm}
$\forall A \in \mathcal{M}: I \leq A$ \quad \mbox{und} \quad
$\forall T \in \mathcal{D}: \Bigl(\bigl(\forall A \in \mathcal{M}: T \leq A \bigr)\rightarrow T \leq I \Bigr)$
\\[0.2cm]
gilt.  In diesem Fall schreiben wir
\\[0.2cm]
\hspace*{1.3cm}
$S = \inf( \mathcal{M} )$.
\end{Definition}

\exercise
Zeigen Sie, dass jede nicht-leere und in $\mathcal{D}$ nach unten beschr�nkte Menge
$\mathcal{M} \subseteq \mathcal{D}$ ein Infimum $I \in \mathcal{D}$ hat.

\begin{Definition}[Addition von Dedekind-Mengen]
Es seien $A$ und $B$ Dedekind-Mengen.  Dann definieren wir die Summe $A + B$ wie folgt:
\\[0.2cm]
\hspace*{1.3cm}
$A + B := \{ x + y \mid x \in A \wedge y \in B \}$. 
\end{Definition}

\exercise
Es seien $A,B \in \mathcal{D}$.  Zeigen Sie, dass dann auch $A + B \in \mathcal{D}$ ist.

\exercise
Zeigen Sie, dass die Menge 
\\[0.2cm]
\hspace*{1.3cm}
$O := \{ x \in \mathbb{Q} \mid x < 0 \}$
\\[0.2cm]
eine Dedekind-Menge ist und zeigen Sie weiter, dass die Struktur $\langle \mathcal{D}, 0, + \rangle$
eine kommutative Gruppe ist.
\pagebreak

\exercise
�berlegen Sie, wie sich auf der Menge $\mathcal{D}$ eine Multiplikation definieren l�sst,
so dass $\mathcal{D}$ mit dieser Multiplikation und der oben definierten Addition ein K�rper wird.
\vspace*{0.5cm}


\noindent
\textbf{Bemerkung}:  Da Dedekind-Mengen und Dedekind-Schnitte sich eins-zu-eins entsprechen,
k�nnen wir  die Menge $\mathbb{R}$ der reellen Zahlen auch als die Menge
$\mathcal{D}$ der Dedekind-Mengen auffassen.   Jedes $A \in \mathcal{D}$ stellt dabei die reelle Zahl 
$\inf(\mathbb{Q} \backslash A)$ dar.  In dem Buch ``\emph{Grundlagen der Analysis}'' von Edmund
Landau \cite{landau:1930} wird diese Konstruktion der reellen Zahlen im Detail beschrieben.  


\begin{Theorem}[Vollst�ndigkeit der reellen Zahlen]
Ist $\pair(M_1,M_2)$ ein Dedekind-Schnitt, so gibt es eine \emph{Trennungs-Zahl} $s\in\mathbb{R}$, 
so dass gilt:
\\[0.2cm]
\hspace*{1.3cm}
$ \forall x_1 \in M_1: x_1 \leq s \quad \mbox{\textrm{und}} \quad
   \forall x_2 \in M_2: s \leq x_2.
$ \qed
\end{Theorem}

\proof
Da reelle Zahlen nichts anderes als Dedekind-Schnitte sind, k�nnen wir ganz einfach
\\[0.2cm]
\hspace*{1.3cm}
$s = \pair(M_1,M_2)$
\\[0.2cm]
definieren.  \qed


\remark
Die Vollst�ndigkeit der reellen Zahlen kann anschaulich als die Aussage interpretiert werden, dass
die Menge der reellen Zahlen keine L�cher hat.  Diese Erkenntnis geht auf Richard Dedekind
\cite{dedekind:1901} zur�ck.

%%% Local Variables: 
%%% mode: latex
%%% TeX-master: "analysis"
%%% End: 

