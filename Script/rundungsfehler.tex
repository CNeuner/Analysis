\chapter{Rundungsfehler$^*$}
Die meisten komplexen Probleme lassen sich nur mit numerischen Verfahren l�sen.  Wir haben
bereits verschiedene numerische Verfahren kennengelernt, beispielsweise Verfahren zur
Berechnung von Nullstellen sowie Verfahren zur numerischen Integration.  Allen diesen
Verfahren ist gemeinsam, dass zwei verschiedene Arten von Fehlern auftreten:
\begin{enumerate}
\item Ein \emph{Approximations-Fehler} tritt auf, wenn wir einen Wert $\lambda$ berechnen
      wollen, zu dessen Berechnung wir nur eine N�herungsformel existiert.  Oft ist der
      Approximations-Fehler ein \emph{Abbruch-Fehler}, der seine
      Ursache darin hat, dass wir nur endlich viele Glieder einer unendlichen Reihe
      berechnen k�nnen.  Wollen wir beispielsweise die Euler'sche Zahl $e$ nach der Formel
      \\[0.2cm]
      \hspace*{1.3cm}
      $e = \sum\limits_{k=0}^\infty \bruch{1}{k!}$
      \\[0.2cm]
      berechnen, so k�nnen wir auf dem Rechner diese Summe nicht gegen unendlich laufen
      lassen, sondern m�ssen die Summe nach endlich vielen Gliedern abbrechen, wir berechnen
      also als Approximation f�r $e$ eine Reihe der Form
      \\[0.2cm]
      \hspace*{1.3cm}
      $e_n := \sum\limits_{k=0}^n \bruch{1}{k!}$ 
      \\[0.2cm]
      und m�ssen dann $n$ so gro� w�hlen, dass der Abbruch-Fehler
      \\[0.2cm]
      \hspace*{1.3cm}
      $e - e_n = \sum\limits_{k=n+1}^\infty \bruch{1}{k!}$ 
      \\[0.2cm]
      unterhalb einer vorgegeben Schranke bleibt.
\item Zus�tzlich zum Approximations-Fehler gibt es noch die Rundungsfehler, die im Laufe der Rechnung 
      entstehen.  Diese Rundungsfehler haben Ihre Ursache darin, dass Flie�komma-Zahlen auf dem Rechner 
      mit einer vorgegebenen Genauigkeit dargestellt werden.  Rechnen wir in der Sprache \textsl{Java}
      mit einer Flie�komma-Zahl vom Typ \texttt{float}, so stehen zur Darstellung der Stellen hinter dem
      Komma lediglich 23 Bits zur Verf�gung.  Werden nun zwei solche Zahlen multipliziert, so k�nnten
      bis zu 47 Bits notwendig sein, um alle Stellen hinter dem Komma korrekt wiedergeben zu k�nnen.
      Da aber zum Abspeichern des Ergebnisses lediglich 23 Bits zur Verf�gung stehen, um die Ziffern 
      hinter dem Komma abzuspeichern, bleibt nichts anderes �brig, als das Ergebnis auf 23 Bits zu
      runden.  Der dadurch entstehende Fehler wird als Rundungsfehler bezeichnet.
\end{enumerate}
Die Auswirkungen von Rundungsfehlern werden oft untersch�tzt.  Unter
\\[0.2cm]
\hspace*{1.3cm}
\href{http://www.devtopics.com/20-famous-software-disasters-part-2/}{\texttt{http://www.devtopics.com/20-famous-software-disasters-part-2/}}
\\[0.2cm]
findet sich eine Liste der 20 spektakul�rsten Software-Fehler, die Katastrophen ausgel�st haben.
In mehreren F�llen waren Rundungsfehler ein Teil des Problems.  Um einen ersten Eindruck von der Wirkung
von Rundungsfehlern zu bekommen, betrachten wir das in Abbildung \ref{fig:harmonic.stlx} gezeigte
Programm zur Berechnung der Reihe
\\[0.2cm]
\hspace*{1.3cm}
$\sum\limits_{n=1}^\infty \bruch{1}{n}$.

\begin{figure}[!ht]
\centering
\begin{Verbatim}[ frame         = lines, 
                  framesep      = 0.3cm, 
                  firstnumber   = 1,
                  labelposition = bottomline,
                  numbers       = left,
                  numbersep     = -0.2cm,
                  xleftmargin   = 0.8cm,
                  xrightmargin  = 0.8cm,
                ]
    harmonic := procedure() {
        oldSum := 0.0;
        sum := 1.0;
        n := 1;
        while (oldSum < sum) {
            oldSum := sum;
            n += 1;
            sum += 1/n;
        }
        print("sum = $sum$, n = $n$");
    };    
    harmonic();
\end{Verbatim}
\vspace*{-0.3cm}
\caption{Berechnung von $\sum\limits_{n=1}^\infty \bruch{1}{n}$.}
\label{fig:harmonic.stlx}
\end{figure}

Sie erwarten jetzt vielleicht, dass diese Programm nie terminiert, aber wenn wir dieses Programm
in einer Datei mit dem Namen ``\texttt{harmonic.stlx}'' speichern und es dann mit dem Befehl
\\[0.2cm]
\hspace*{1.3cm}
\texttt{setlX --real32 harmonic.stlx}
\\[0.2cm]
starten, dann erhalten wir nach wenigen Sekunden die Meldung:
\\[0.2cm]
\hspace*{1.3cm}
\texttt{sum = 13.05426, n = 200001}.
\\[0.2cm]
Da wir fr�her bewiesen haben, dass die Partialsummen $\sum\limits_{k=1}^n \bruch{1}{k}$ f�r wachsende Werte
von $n$ beliebig gro� werden, fragen wir uns, was bei der Rechnung schief gelaufen ist.
Die Antwort ist, dass f�r $n= 200001$ der Wert $\frac{1}{n}$ so klein ist, dass die Summe
\\[0.2cm]
\hspace*{1.3cm}
$13.05426 + \frac{1}{n}$
\\[0.2cm]
so nahe bei $13.05426$ liegt, dass sie auf den Wert $13.05426$ abgerundet wird.  Um diesen Effekt n�her
zu beschreiben, definiert man f�r einen vorgegebenen Rechner die sogenannte
\emph{Maschinen-Konstante} \textsl{eps} als die kleinste positive Zahl, die, wenn sie auf diesem Rechner zu
$1$ addiert wird, eine Ergebnis gr��er als 1 ergibt.  Die formale Definition lautet
\\[0.2cm]
\hspace*{1.3cm}
$\textsl{eps} := \min\bigl( \{ x \in \mathbb{R} \mid x > 0 \wedge 1 \oplus x > 1 \} \bigr)$.
\\[0.2cm]
Hier bezeichnet $\oplus$ die auf dem Rechner implementierte Addition.
Abbildung \ref{fig:maschinen-konstante.stlx} zeigt ein einfaches Programm zur Berechnung der
Maschinen-Konstante.  Bei der im \textsc{Ieee}-Standard 754 definierten 32-Bit-Architektur erhalten wir
f�r $\textsl{eps}$ den Wert
\\[0.2cm]
\hspace*{1.3cm}
$\textsl{eps}_{32} = 9.536745 \cdot 10^{-7}$,
\\[0.2cm]
bei einer 64-Bit-Architektur lautet das Ergebnis
\\[0.2cm]
\hspace*{1.3cm}
$\textsl{eps}_{64} = 8.88178419700125 \cdot 10^{-16}$,
\\[0.2cm]
und wenn wir mit 128-Bit rechnen, haben wir
\\[0.2cm]
\hspace*{1.3cm}
$\textsl{eps}_{128} = 7.703719777548943412223911770339695 \cdot 10^{-34}$.
\vspace*{0.3cm}

\begin{figure}[!ht]
\centering
\begin{Verbatim}[ frame         = lines, 
                  framesep      = 0.3cm, 
                  firstnumber   = 1,
                  labelposition = bottomline,
                  numbers       = left,
                  numbersep     = -0.2cm,
                  xleftmargin   = 0.8cm,
                  xrightmargin  = 0.8cm,
                ]
    maschinenKonstante := procedure() {
        eps := 1.0;
        old := eps;
        while (1.0 + eps > 1.0) {
            old := eps;
            eps /= 2;
        }
        return old;
    };
\end{Verbatim}
\vspace*{-0.3cm}
\caption{Berechnung der Maschinen-Konstante \textsl{eps}.}
\label{fig:maschinen-konstante.stlx}
\end{figure}

\noindent
Bei modernen Rechnern, die den \textsc{Ieee}-Standard 754 implementieren, k�nnen wir davon ausgehen, dass
der relative Rundungsfehler bei der Ausf�hrung 
einer  Grundrechenoperation durch die Maschinen-Konstante \textsl{eps} beschr�nkt ist.

Leider muss die Vorlesung aus Zeitgr�nden an dieser Stelle enden.  Der interessierten Leser sei daher
auf die Literatur, insbesondere den Artikel von Goldberg \cite{goldberg:91} verwiesen. 

%%% Local Variables: 
%%% mode: latex
%%% TeX-master: "analysis"
%%% End: 
