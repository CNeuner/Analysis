\chapter{Einleitung}
Der vorliegende Text ist ein Fragment eines Vorlesungs-Skriptes f�r die Analysis-Vorlesung f�r
Informatiker.  Ich habe mich bei der Ausarbeitung dieser Vorlesung im wesentlichen auf die folgenden
Quellen gest�tzt:
\begin{enumerate}
\item Analysis \texttt{I} von Otto Forster \cite{forster:2011}.
\item Differential- und Integralrechnung \texttt{I} von Hans Grauert und Ingo Lieb \cite{grauert:1967}.
\item Lehrbuch der Analysis, Teil 1 und Teil 2 von Harro Heuser \cite{heuser:2003,heuser:2008}.
\end{enumerate}
Den Studenten empfehle ich das erste Buch in dieser Liste, denn dieses Buch ist auch in
elektronischer Form in unserer Bibliothek vorhanden. 

\section{�berblick �ber die Vorlesung}
Im Rahmen der Vorlesung werden die folgenden Gebiete behandelt:
\begin{enumerate}
\item Das zweite Kapitel f�hrt den Begriff des Grenzwerts f�r Folgen und Reihen ein.
\item Das dritte Kapitel diskutiert die Begriffe Stetigkeit und Differenzierbarkeit.
\item Das vierte Kapitel zeigt verschiedene Anwendungen der bis dahin dargestellten Theorie.
      Insbesondere werden \emph{Taylor-Reihen} diskutiert. Diese k�nnen beispielsweise zur
      Berechnung der trigonometrischen Funktionen verwendet werden.  Au�erdem diskutieren wir in
      diesem Kapitel Verfahren zur numerischen L�sung von Gleichungen.
\item Das f�nfte Kapitel besch�ftigt sich mit der Integralrechnung.
\item Im sechsten Kapitel zeigen wir, dass  $\pi$ und $e$ keine rationalen Zahlen sind.
\item Im letzten Kapitel diskutieren wir Fourier-Reihen.
\end{enumerate}

%%% Local Variables: 
%%% mode: latex
%%% TeX-master: "analysis"
%%% End: 
